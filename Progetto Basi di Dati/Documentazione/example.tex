\documentclass[oneside]{book}
\usepackage[margin=3cm]{geometry}
\usepackage[utf8]{inputenc}
\usepackage[T1]{fontenc}
\usepackage[italian]{babel}
\usepackage{graphicx}
\usepackage{listings}
\usepackage{xcolor}




% Configurazione SQL
\lstdefinestyle{sqlstyle}{
    language=SQL,
    basicstyle=\ttfamily\small,
    keywordstyle=\color{blue}\bfseries,
    commentstyle=\color{green!60!black},
    stringstyle=\color{red},
    numberstyle=\tiny\color{gray},
    numbers=left,
    numbersep=5pt,
    frame=single,
    breaklines=true,
    breakatwhitespace=true,
    tabsize=2,
    showspaces=false,
    showstringspaces=false,
    captionpos=b
}

\lstset{style=sqlstyle}

% Configurazione per caratteri accentati
\lstset{
    literate={à}{{\`a}}1 {è}{{\`e}}1 {ì}{{\`i}}1 {ò}{{\`o}}1 {ù}{{\`u}}1
             {À}{{\`A}}1 {È}{{\`E}}1 {Ì}{{\`I}}1 {Ò}{{\`O}}1 {Ù}{{\`U}}1
             {á}{{\'a}}1 {é}{{\'e}}1 {í}{{\'i}}1 {ó}{{\'o}}1 {ú}{{\'u}}1
             {Á}{{\'A}}1 {É}{{\'E}}1 {Í}{{\'I}}1 {Ó}{{\'O}}1 {Ú}{{\'U}}1
}

% STEP 1: include the package
\usepackage{uninafrontespizio}

% STEP 2: Front-page configuration
\Universita{Università degli Studi di Napoli Federico II}
\Facolta{Scuola Politecnica e delle Scienze di Base}
\Dipartimento{Dipartimento di Ingegneria Elettrica e Tecnologie dell'Informazione}
\CorsoDiLaurea{Corso di Laurea Triennale in Informatica}
% \Materia{Tesi di Laurea in Hacky Typesetting} % optional
\AnnoAccademico{Anno Accademico 2024--2025}
\Titolo{UninaFoodLab Database}

\Logo{logo-federico-II.pdf} %path to logo image
\LogoWidth{3.5cm} %optional, default: 3cm
\LogoPosition{below-uni} % or top, or below-title, or above-title, or no-logo

\begin{document}
    % STEP 3: use \makefrontpage and/or \makefrontpagealt
    \pagestyle{empty}
    \makefrontpage%
    % \makefrontpagealt
    \pagestyle{plain}
    \tableofcontents

    \chapter{Descrizione del progetto}
    \section{Descrizione sintetica del problema}
    Si vuole realizzare una base di dati relazionale per la gestione di corsi di cucina tematici.
    Gli chef possono registrare corsi su specifici argomenti, specificando una data di inizio e
    una frequenza delle sessioni. Ogni corso è articolato in più sessioni,
    che possono essere di due tipi: online, oppure in presenza, in cui gli utenti svolgono attività pratiche.
    Gli utenti possono iscriversi a più corsi e, nel caso delle sessioni pratiche, devono fornire una adesione
    esplicita per confermare la loro partecipazione. Ogni sessione pratica prevede la preparazione di una o
    più ricette, ciascuna delle quali richiede una specifica lista di ingredienti. Le adesioni vengono utilizzate
    per pianificare correttamente la quantità di ingredienti necessari, evitando così sprechi alimentari. 
    Il sistema prevede la possibilità di notificare gli utenti iscritti ad
    un corso in caso di variazioni di programma.
    
    \chapter{Progettazione Concettuale}
    \section{Introduzione}
    Una volta definito e analizzato il problema possiamo procedere con la progettazione concettuale della base di dati. In questa fase, 
    l'obiettivo è quello di creare un modello concettuale che rappresenti le entità, le relazioni e le caratteristiche del dominio. Tale schema viene
    rappresentato mediante un diagramma delle classi UML e un diagramma Entità-Relazioni (ER).    
    \section{Diagrammi UML e ER}
    \begin{figure}[h]
        \centering
        \includegraphics[width=\textwidth]{../Diagramma UML/UML.png}
        \caption{Diagramma delle classi UML}
        \label{fig:uml}
    \end{figure}
    \begin{figure}[h]
        \centering
        \includegraphics[width=\textwidth]{../Diagramma ER/ER.png}
        \caption{Diagramma Entità-Relazioni}
        \label{fig:er}
    \end{figure}
    \section{Ristrutturazione del modello concettuale}
    Un'operazione fondamentale per il passaggio dal modello concettuale a quello logico è la ristrutturazione. Questo processo 
    implica la trasformazione delle entità e delle relazioni del modello concettuale, al fine di renderne più agevole l'implementazione 
    in un database relazionale. Durante questa fase, si identificano le chiavi primarie e le chiavi esterne, si rimuovono gli attributi multipli
    e le gerarchie di specializzazione, si normalizzano le tabelle per ridurre la ridondanza dei dati.
    \section{Identificazione delle chiavi primarie}
    Le chiavi primarie sono gli attributi che identificano in modo univoco ogni record in una tabella. Nella nostra ristrutturazione
    abbiamo fatto uso di chiavi surrogate, ossia chiavi artificiali create per garantire l'unicità dei record nel caso in cui non esistano 
    chiavi naturali adatte. Queste chiavi non hanno un significato intrinseco, ma sono utilizzate per facilitare le operazioni sulle tabelle.
    \section{Identificazione delle chiavi esterne}
    Le chiavi esterne sono gli attributi che collegano le tabelle tra loro, rappresentando le relazioni tra le entità. Il loro scopo 
    è quello di assicurare che i dati siano coerenti e che le relazioni siano mantenute.
    \section{Rimozione degli attributi multipli}
    Non sono presenti attributi multipli nel modello concettuale.
    \section{Rimozione delle gerarchie di specializzazione}
    Sono presenti due gerarchie di specializzazione:
    \begin{itemize}
        \item 'Persona' si specializza in "Chef" e "Utente"
        \item 'Sessione' si specializza in "Sessione Online" e "Sessione in Presenza"
    \end{itemize}
    \section{Analisi delle ridondanze}
    Non sono presenti ridondanze significative nel modello concettuale.
    \section{Class Diagram ristrutturato}
    \begin{figure}[h]
        \centering
        \includegraphics[width=\textwidth]{../Diagramma UML Ristrutturato/UMLRistrutturato.png}
        \caption{Diagramma UML ristrutturato}
        \label{fig:uml-ristrutturato}
    \end{figure}

    \newpage
    \section{Dizionario delle classi}

    \begin{table}[h]
        \centering
        \begin{tabular}{|p{3cm}|p{4cm}|p{8cm}|}
            \hline
            \textbf{Classe} & \textbf{Descrizione} & \textbf{Attributi} \\
            \hline
            Chef & Rappresenta gli chef della piattaforma, possono tenere corsi sugli argomenti di cui sono esperti & Username (string), Nome (string), Cognome (string), Password (string) \\
            \hline
            Utente & Rappresenta gli utenti della piattaforma, possono iscriversi ai corsi & Username (string), Nome (string), Cognome (string), Password (string) \\
            \hline
            Topic & Rappresenta gli argomenti dei corsi e le conoscenze degli chef & IdTopic (string), Nome (string), Descrizione (string) \\
            \hline
            Corso & Rappresenta i corsi tenuti dagli chef e ai quali gli utenti possono iscriversi & IdCorso (string), Titolo (string), Frequenza (string), NumLezioni (integer), Anno(integer), DataInizio (date) \\
            \hline
            Iscrizione & Rappresenta l'iscrizione di un utente a un corso & DataIscrizione (date) \\
            \hline
            Notifica & Rappresenta le notifiche che gli chef possono inviare agli utenti & IdNotifica (string), DataInvio (date), Oggetto (string), Testo (string) \\
            \hline
            SessioneOnline & Rappresenta le sessioni online dei corsi & IdSessione (string), Link (string), Data (date), Durata (integer), NumSessione (integer) \\
            \hline
            SessioneInPresenza & Rappresenta le sessioni in presenza dei corsi & IdSessione (string), Luogo (string), Aula (string), Data (date), Durata (integer), NumSessione (integer) \\
            \hline
            Ricetta & Rappresenta le ricette che possono essere preparate durante le sessioni in presenza & IdRicetta (string), Nome (string), Descrizione (string) \\
            \hline
            Ingrediente & Rappresenta gli ingredienti utilizzabili per le ricette & IdIngrediente (string), Nome (string), UnitàDiMisura (string), Allergene (boolean) \\
            \hline
            Composizione & Rappresenta le quantità di ogni ingrediente in una ricetta & Quantità(float) \\
            \hline
        \end{tabular}
    \end{table}

    \newpage
    \section{Dizionario delle associazioni}

    \begin{table}[h]
        \centering
        \begin{tabular}{|p{3cm}|p{4cm}|p{8cm}|}
            \hline
            \textbf{Associazione} & \textbf{Descrizione} & \textbf{Classi coinvolte} \\
            \hline
            Insegna & Associa uno chef a un topic che insegna & Chef[0,*], Topic[1,*] \\
            \hline
            Riceve & Associa un utente a una notifica ricevuta & Utente[1,*], Notifica[0,*] \\
            \hline
            È iscritto a & Associa un utente a un corso a cui è iscritto & Utente[0,*], Corso[0,*], Iscrizione (Classe di associazione) \\
            \hline
            Aderisce a & Associa un utente a una sessione in presenza a cui partecipa & Utente[1,*], SessioneInPresenza[0,*] \\
            \hline
            Si prepara & Associa una sessione in presenza a una ricetta preparata in quella sessione & SessioneInPresenza[0,*], Ricetta[1,*] \\
            \hline
            È composta da & Associa una ricetta a un ingrediente che la compone & Ricetta[0,*], Ingrediente[1,*], Composizione (Classe di associazione) \\
            \hline
            Tiene & Associa uno chef a un corso che insegna & Chef[1], Corso[0,*] \\
            \hline
            Afferisce a & Associa un corso al topic di appartenenza & Corso[0,*], Topic[1] \\
            \hline
            È composto da (Online) & Associa un corso alle sue sessioni online & Corso[1], SessioneOnline[1,*] \\
            \hline
            È composto da (Presenza) & Associa un corso alle sue sessioni in presenza & Corso[1], SessioneInPresenza[1,*] \\
            \hline
            Invia & Associa uno chef alle notifiche che invia & Chef[1], Notifica[0,*] \\
            \hline
        \end{tabular}
    \end{table}

    \newpage
    \section{Dizionario dei vincoli}
    \begin{table}[h]
        \centering
        \begin{tabular}{|p{4cm}|p{11cm}|}
            \hline
            \textbf{Nome} & \textbf{Descrizione} \\
            \hline
            Validità Anno & Questo vincolo viene attivato prima dell'inserimento o aggiornamento di un nuovo corso e verifica che l'anno del corso sia l'anno corrente o il successivo. Se l'anno non è valido, il trigger genera un errore e impedisce l'inserimento del corso. Questo è utile per garantire che i corsi siano pianificati correttamente e che le date siano coerenti con l'anno accademico corrente o successivo. \\
            \hline
            Aula Occupata & Questo vincolo viene attivato prima dell'inserimento o aggiornamento di una sessione in presenza e verifica che l'aula specificata non sia già occupata da un'altra sessione in presenza nello stesso orario. Se l'aula è occupata, il trigger genera un errore e impedisce l'inserimento della sessione. Questo è utile per evitare conflitti di programmazione e garantire che le sessioni in presenza possano essere svolte senza problemi logistici. \\
            \hline
            Iscrizione Per Adesione & Questo vincolo viene attivato prima dell'inserimento di una nuova adesione e verifica che l'utente che sta aderendo a una sessione in presenza sia iscritto al corso corrispondente. Se l'utente non è iscritto al corso, il trigger genera un errore e impedisce l'inserimento dell'adesione. Questo è utile per garantire che solo gli utenti iscritti ai corsi possano partecipare alle sessioni in presenza, mantenendo la coerenza dei dati e le regole del sistema. \\
            \hline
            Insegnamenti Chef & Questo vincolo viene attivato prima dell'inserimento di un nuovo corso e verifica che lo chef che sta creando il corso insegni effettivamente il topic associato a questo. Se lo chef non insegna il topic specificato, il trigger genera un errore e impedisce l'inserimento del corso. Questo è utile per garantire che i corsi siano tenuti da chef competenti nei rispettivi argomenti, mantenendo la qualità dell'insegnamento. \\
            \hline
            Data Sessione Corso & Questo vincolo viene attivato prima dell'inserimento o aggiornamento di una sessione e verifica che la data della sessione sia successiva alla data di inizio del corso a cui appartiene. Se la data della sessione non è valida, il trigger genera un errore e impedisce l'inserimento della sessione. Questo è utile per garantire che le sessioni siano programmate correttamente in relazione al corso, evitando conflitti di programmazione e garantendo che le sessioni si svolgano dopo l'inizio del corso stesso. \\
            \hline
            Data Inizio Corso & Questo vincolo viene attivato prima dell'inserimento o aggiornamento di un nuovo corso e verifica che la data di inizio del corso sia una data futura. Se la data non è valida, il trigger genera un errore e impedisce l'inserimento del corso. Questo è utile per garantire che i corsi siano pianificati correttamente e che le date siano coerenti. \\
            \hline
        \end{tabular}
    \end{table}

    \begin{table}[h]
        \centering
        \begin{tabular}{|p{4cm}|p{11cm}|}
            \hline
            \textbf{Nome} & \textbf{Descrizione} \\
            \hline
            Limite Iscrizione & Questo vincolo viene attivato prima dell'inserimento di una nuova iscrizione e verifica che il corso a cui l'utente sta cercando di iscriversi non abbia già raggiunto il limite di 50 iscrizioni. Se il limite è stato raggiunto, il trigger genera un errore e impedisce l'inserimento dell'iscrizione. Questo è utile per garantire che i corsi non superino il numero massimo di partecipanti, mantenendo la qualità dell'insegnamento e la gestione delle risorse. \\ 
            \hline
            Numero Sessione & Questo vincolo viene attivato prima dell'inserimento di una sessione e verifica che il numero della sessione sia coerente con il numero della sessione precedente. Se il numero della sessione non è valido, il trigger genera un errore e impedisce l'inserimento della sessione. Questo è utile per garantire che le sessioni siano programmate in modo sequenziale e che non ci siano salti nei numeri delle sessioni, mantenendo la coerenza del sistema. \\
            \hline
            Sovrapposizioni Date & Questo vincolo viene attivato prima dell'inserimento di una sessione e verifica che non ci sia una sovrapposizione di date tra le sessioni per lo stesso chef. Se c'è una sovrapposizione, il trigger genera un errore e impedisce l'inserimento della sessione. Questo è utile per evitare conflitti di programmazione e garantire che gli chef possano gestire le loro sessioni in presenza senza problemi logistici. \\
            \hline
            Unicità Username Utenti Chef & Questo vincolo viene attivato prima dell'inserimento o aggiornamento di un nuovo chef o utente e verifica che lo username sia univoco. Se lo username è già presente nel database, il trigger genera un errore e impedisce l'inserimento o l'aggiornamento. Questo è utile per garantire che gli username degli chef e degli utenti siano univoci, evitando conflitti e garantendo l'integrità dei dati. \\
            \hline
            Data Sessione & Questo vincolo viene attivato prima dell'inserimento o aggiornamento di una sessione e verifica che la data della sessione sia una data futura. Se la data non è valida, il trigger genera un errore e impedisce l'inserimento della sessione. Questo è utile per garantire che le sessioni siano programmate correttamente e che le date siano coerenti. \\
            \hline
            CK Topic Nome & Questo vincolo controlla che il nome del topic sia compreso in un elenco di cucine disponibili. \\
            \hline
            CK Frequenza & Questo vincolo controlla che la frequenza del corso sia compresa in un elenco di frequenze valide. \\
            \hline
            CK NumeLezioni & Questo vincolo controlla che il numero di lezioni di un corso sia compreso tra 1 e 100. \\
            \hline
            CK Anno & Questo vincolo controlla che l'anno di un corso corrisponda a quello dell'inizio del corso stesso. \\
            \hline
            CK Luogo & Questo vincolo controlla che il luogo in cui è tenuta la sessione in presenza sia compreso in un elenco di luoghi validi. \\
            \hline
            CK Durata & Questo vincolo controlla che la durata di una sessione, online o in presenza, sia compresa tra 60 e 180 minuti. \\
            \hline
            CK Link & Questo vincolo controlla che il link abbia un formato valido. \\
            \hline
            CK Quantità Positiva & Questo vincolo controlla che la quantità di un ingrediente in una ricetta sia un valore positivo. \\
            \hline
        \end{tabular}
    \end{table}

    \chapter{Progettazione Logica}
    \section{Schema Logico}
    \textbf{TOPIC}(\underline{IdTopic}, Nome, Descrizione)\\\\
    \textbf{INSEGNA}(\underline{\underline{IdTopic}}, \underline{\underline{UsernameChef}})\\\\
    \textbf{CORSO}(\underline{IdCorso}, Titolo, Frequenza, NumLezioni, DataInizio, \underline{\underline{IdTopic}}, \underline{\underline{UsernameChef}})\\\\
    \textbf{CHEF}(\underline{Username}, Nome, Cognome, Password)\\\\
    \textbf{UTENTE}(\underline{Username}, Nome, Cognome, Password)\\\\
    \textbf{NOTIFICA}(\underline{IdNotifica}, DataInvio, Oggetto, Testo, \underline{\underline{UsernameChef}})\\\\
    \textbf{RICEVE}(\underline{\underline{IdNotifica}}, \underline{\underline{UsernameUtente}})\\\\
    \textbf{ISCRIZIONE}(\underline{\underline{UsernameUtente}}, \underline{\underline{IdCorso}}, DataIscrizione)\\\\
    \textbf{SESSIONE ONLINE}(\underline{IdSessione}, Link, Data, Durata, NumSessione, \underline{\underline{IdCorso}})\\\\
    \textbf{SESSIONE IN PRESENZA}(\underline{IdSessione}, Luogo, Aula, Data, Durata, NumSessione, \underline{\underline{IdCorso}})\\\\
    \textbf{ADESIONE}(\underline{\underline{UsernameUtente}}, \underline{\underline{IdSessionePresenza}})\\\\
    \textbf{RICETTA}(\underline{IdRicetta}, Nome, Descrizione)\\\\
    \textbf{PREPARA}(\underline{\underline{IdSessionePresenza}}, \underline{\underline{IdRicetta}})\\\\
    \textbf{INGREDIENTE}(\underline{IdIngrediente}, Nome, UnitàDiMisura, Allergene)\\\\
    \textbf{COMPOSIZIONE}(\underline{\underline{IdRicetta}}, \underline{\underline{IdIngrediente}}, Quantità)\\\\
    \chapter{Progettazione Fisica}
    La base di dati è stata implementata utilizzando PostgreSQL, un sistema di gestione di basi di dati relazionali open source.
    La progettazione fisica della base di dati è stata effettuata seguendo le linee guida del modello relazionale, 
    definendo le tabelle, le chiavi primarie e le chiavi esterne per garantire l'integrità referenziale tra le entità.
    Qui di seguito sono riportate le definizioni delle tabelle utilizzate nel database ed eventuali triggers, procedures, functions o vincoli implementati.
    \section{Definizione tabelle}
    Seguono le definizioni delle tabelle estratte dallo script di creazione del database.
    \subsection{Definizione della tabella TOPIC}
    \lstinputlisting[caption=Creazione tabella Topic]{../SQL/Tabelle/Topic.sql}

    \subsection{Definizione della tabella CHEF}
    \lstinputlisting[caption=Creazione tabella Chef]{../SQL/Tabelle/Chef.sql}

    \subsection{Definizione della tabella UTENTE}
    \lstinputlisting[caption=Creazione tabella Utente]{../SQL/Tabelle/Utente.sql}

    \subsection{Definizione della tabella CORSO}
    \lstinputlisting[caption=Creazione tabella Corso]{../SQL/Tabelle/Corso.sql}

    \subsection{Definizione della tabella SESSIONEINPRESENZA}
    \lstinputlisting[caption=Creazione tabella Sessione in Presenza]{../SQL/Tabelle/SessioneInPresenza.sql}
    
    \subsection{Definizione della tabella SESSIONEONLINE}
    \lstinputlisting[caption=Creazione tabella Sessione Online]{../SQL/Tabelle/SessioneOnline.sql}
    
    \subsection{Definizione della tabella ADESIONE}
    \lstinputlisting[caption=Creazione tabella Adesione]{../SQL/Tabelle/Adesione.sql}
    
    \subsection{Definizione della tabella PREPARA}
    \lstinputlisting[caption=Creazione tabella Prepara]{../SQL/Tabelle/Prepara.sql}
    
    \subsection{Definizione della tabella RICETTA}
    \lstinputlisting[caption=Creazione tabella Ricetta]{../SQL/Tabelle/Ricetta.sql}
    
    \subsection{Definizione della tabella INGREDIENTE}
    \lstinputlisting[caption=Creazione tabella Ingrediente]{../SQL/Tabelle/Ingrediente.sql}
    
    \subsection{Definizione della tabella COMPOSIZIONE}
    \lstinputlisting[caption=Creazione tabella Composizione]{../SQL/Tabelle/Composizione.sql}
    
    \subsection{Definizione della tabella NOTIFICA}
    \lstinputlisting[caption=Creazione tabella Notifica]{../SQL/Tabelle/Notifica.sql}
    
    \subsection{Definizione della tabella RICEVE}
    \lstinputlisting[caption=Creazione tabella Riceve]{../SQL/Tabelle/Riceve.sql}
   
    \subsection{Definizione della tabella INSEGNA}
    \lstinputlisting[caption=Creazione tabella Insegna]{../SQL/Tabelle/Insegna.sql}
    
    \subsection{Definizione della tabella ISCRIZIONE}
    \lstinputlisting[caption=Creazione tabella Iscrizione]{../SQL/Tabelle/Iscrizione.sql}
  
    \section{Triggers, Procedures, Functions e Vincoli}
    Durante la progettazione fisica della base di dati, sono stati implementati alcuni triggers, procedures, functions e vincoli per garantire l'integrità dei dati e automatizzare alcune operazioni. Di seguito sono riportati alcuni esempi:
    \subsection{Funzione per calcolare il numero di adesioni}
    \lstinputlisting[caption=Funzione per calcolare il numero di adesioni]{../SQL/Trigger-Procedures-Functions/CalcoloNumeroAdesioni.sql}
    Questa funzione calcola il numero di adesioni per una sessione in presenza. Necessaria per pianificare correttamente la quantità di ingredienti necessari per ogni sessione in presenza.
    
    \newpage
    \subsection{Implementazione del trigger Validità Anno}
    \lstinputlisting[caption=Trigger per controllare l'anno del corso]{../SQL/Trigger-Procedures-Functions/TriggerAnnoCorso.sql}

    \subsection{Implementazione del trigger Aula Occupata}
    \lstinputlisting[caption=Trigger per controllare l'aula della sessione in presenza]{../SQL/Trigger-Procedures-Functions/TriggerAulaOccupata.sql}

    \subsection{Implementazione del trigger Iscrizione Per Adesione}
    \lstinputlisting[caption=Trigger per controllare l'iscrizione dell'utente]{../SQL/Trigger-Procedures-Functions/TriggerControlloAdesioniCorso.sql}

    \newpage
    \subsection{Implementazione del trigger Insegnamenti Chef}
    \lstinputlisting[caption=Trigger per controllare il topic del corso]{../SQL/Trigger-Procedures-Functions/TriggerControlloInsegna.sql}
    
    \subsection{Implementazione del trigger Data Sessione Corso}
    \lstinputlisting[caption=Trigger per controllare le date delle sessioni]{../SQL/Trigger-Procedures-Functions/TriggerControlloSessioneCorsoData.sql}
    
    \subsection{Implementazione del trigger Data Inizio Corso}
    \lstinputlisting[caption=Trigger per controllare la data di inizio del corso]{../SQL/Trigger-Procedures-Functions/TriggerDataInizio.sql}

    \subsection{Implementazione del trigger Data Sessione}
    \lstinputlisting[caption=Trigger per controllare la data delle sessioni]{../SQL/Trigger-Procedures-Functions/TriggerDataSessione.sql}

    \subsection{Implementazione del trigger Limite Iscrizioni}
    \lstinputlisting[caption=Trigger per controllare il limite di iscrizioni al corso]{../SQL/Trigger-Procedures-Functions/TriggerLimiteIscrizioni.sql}

    \subsection{Implementazione del trigger Numero Sessione}
    \lstinputlisting[caption=Trigger per controllare il numero della sessione]{../SQL/Trigger-Procedures-Functions/TriggerNumSessione.sql}
    
    \subsection{Implementazione del trigger Sovrapposizione Date}
    \lstinputlisting[caption=Trigger per controllare la sovrapposizione di date tra le sessioni in presenza]{../SQL/Trigger-Procedures-Functions/TriggerSovrapposizioniDate.sql}

    \subsection{Implementazione del trigger Unicità Username Utenti Chef}
    \lstinputlisting[caption=Trigger per controllare l'unicità degli username]{../SQL/Trigger-Procedures-Functions/TriggerUsernameUtentiChef.sql}
  

    \chapter{Popolamento del database}
    \section{Inserimento dei dati}
    Il database è stato popolato con dati di esempio per testare le funzionalità e verificare il corretto funzionamento delle query. Sono stati inseriti corsi, sessioni, adesioni, ricette e ingredienti, in modo da simulare un ambiente di apprendimento realistico. Di seguito sono riportati alcuni esempi di query utilizzate per il popolamento del database.\\\\
    \subsection{Query di inserimento degli chef}
    \lstinputlisting[caption=Query di inserimento degli chef]{../SQL/Insert/InsertChef.sql}
    Questa query inserisce gli chef nel database, specificando il loro username, nome, cognome e password. Gli chef sono i responsabili dei corsi di cucina.\\\\

    \subsection{Query di inserimento degli utenti}
    \lstinputlisting[caption=Query di inserimento degli utenti]{../SQL/Insert/InsertUtente.sql}
    Questa query inserisce gli utenti nel database, specificando il loro username, nome, cognome e password. Gli utenti sono coloro che si iscrivono ai corsi di cucina.\\\\

    \subsection{Query di inserimento dei corsi}
    \lstinputlisting[caption=Query di inserimento dei corsi]{../SQL/Insert/InsertCorso.sql}
    Questa query inserisce i corsi nel database, specificando il titolo, la frequenza, il numero di lezioni, la data di inizio, il topic e lo chef che insegna il corso. I corsi sono le categorie tematiche dei corsi di cucina.\\\\

    \subsection{Query di inserimento delle sessioni in presenza}
    \lstinputlisting[caption=Query di inserimento delle sessioni in presenza]{../SQL/Insert/InsertSessioneInPresenza.sql}
    Questa query inserisce le sessioni in presenza nel database, specificando il luogo, l'aula, la data, la durata, il numero di sessione e il corso a cui appartiene. Le sessioni in presenza sono le sessioni pratiche dei corsi di cucina.\\\\

    \subsection{Query di inserimento delle sessioni online}
    \lstinputlisting[caption=Query di inserimento delle sessioni online]{../SQL/Insert/InsertSessioneOnline.sql}
    Questa query inserisce le sessioni online nel database, specificando il link, la data, la durata, il numero di sessione e il corso a cui appartiene. Le sessioni online sono le sessioni teoriche dei corsi di cucina.\\\\

    \subsection{Query di inserimento delle adesioni}
    \lstinputlisting[caption=Query di inserimento delle adesioni]{../SQL/Insert/InsertAdesioni.sql}
    Questa query inserisce le adesioni degli utenti alle sessioni in presenza dei corsi di cucina, specificando l'utente e la sessione in presenza a cui si riferiscono.\\\\

    \subsection{Query di inserimento delle ricette}
    \lstinputlisting[caption=Query di inserimento delle ricette]{../SQL/Insert/InsertRicetta.sql}
    Questa query inserisce le ricette nel database, specificando il nome e la descrizione della ricetta. Le ricette sono associate alle sessioni in presenza dei corsi di cucina.\\\\

    \subsection{Query di inserimento degli ingredienti}
    \lstinputlisting[caption=Query di inserimento degli ingredienti]{../SQL/Insert/InsertIngrediente.sql}
    Questa query inserisce gli ingredienti nel database, specificando il nome, l'unità di misura e se è un allergene.\\\\

    \subsection{Query di inserimento della composizione delle ricette}
\begin{lstlisting}[caption=Query di inserimento della composizione delle ricette]
-- Cucina Italiana

-- Spaghetti Aglio Olio e Peperoncino
INSERT INTO Composizione (IdIngrediente, IdRicetta, Quantita)
VALUES
    ((SELECT IdIngrediente FROM Ingrediente WHERE Nome = 'Spaghetti'), 
     (SELECT IdRicetta FROM Ricetta WHERE Nome = 'Spaghetti Aglio Olio e Peperoncino'), 100),
    ((SELECT IdIngrediente FROM Ingrediente WHERE Nome = 'Aglio'), 
     (SELECT IdRicetta FROM Ricetta WHERE Nome = 'Spaghetti Aglio Olio e Peperoncino'), 3),
    ((SELECT IdIngrediente FROM Ingrediente WHERE Nome = 'Olio extravergine d''oliva'), 
     (SELECT IdRicetta FROM Ricetta WHERE Nome = 'Spaghetti Aglio Olio e Peperoncino'), 40),
    ((SELECT IdIngrediente FROM Ingrediente WHERE Nome = 'Peperoncino'), 
     (SELECT IdRicetta FROM Ricetta WHERE Nome = 'Spaghetti Aglio Olio e Peperoncino'), 1),
    ((SELECT IdIngrediente FROM Ingrediente WHERE Nome = 'Prezzemolo'), 
     (SELECT IdRicetta FROM Ricetta WHERE Nome = 'Spaghetti Aglio Olio e Peperoncino'), 10);

-- Pasta e Patate
INSERT INTO Composizione (IdIngrediente, IdRicetta, Quantita)
VALUES
    ((SELECT IdIngrediente FROM Ingrediente WHERE Nome = 'Rigatoni'), 
     (SELECT IdRicetta FROM Ricetta WHERE Nome = 'Pasta e Patate'), 80),
    ((SELECT IdIngrediente FROM Ingrediente WHERE Nome = 'Patate'), 
     (SELECT IdRicetta FROM Ricetta WHERE Nome = 'Pasta e Patate'), 200),
    ((SELECT IdIngrediente FROM Ingrediente WHERE Nome = 'Sedano'), 
     (SELECT IdRicetta FROM Ricetta WHERE Nome = 'Pasta e Patate'), 50),
    ((SELECT IdIngrediente FROM Ingrediente WHERE Nome = 'Carote'), 
     (SELECT IdRicetta FROM Ricetta WHERE Nome = 'Pasta e Patate'), 50),
    ((SELECT IdIngrediente FROM Ingrediente WHERE Nome = 'Cipolla bianca'), 
     (SELECT IdRicetta FROM Ricetta WHERE Nome = 'Pasta e Patate'), 60),
    ((SELECT IdIngrediente FROM Ingrediente WHERE Nome = 'Concentrato di pomodoro'), 
     (SELECT IdRicetta FROM Ricetta WHERE Nome = 'Pasta e Patate'), 20);

-- Carbonara
INSERT INTO Composizione (IdIngrediente, IdRicetta, Quantita)
VALUES
    ((SELECT IdIngrediente FROM Ingrediente WHERE Nome = 'Spaghetti'), 
     (SELECT IdRicetta FROM Ricetta WHERE Nome = 'Carbonara'), 100),
    ((SELECT IdIngrediente FROM Ingrediente WHERE Nome = 'Guanciale'), 
     (SELECT IdRicetta FROM Ricetta WHERE Nome = 'Carbonara'), 50),
    ((SELECT IdIngrediente FROM Ingrediente WHERE Nome = 'Uova di gallina'), 
     (SELECT IdRicetta FROM Ricetta WHERE Nome = 'Carbonara'), 2),
    ((SELECT IdIngrediente FROM Ingrediente WHERE Nome = 'Pecorino Romano'), 
     (SELECT IdRicetta FROM Ricetta WHERE Nome = 'Carbonara'), 40),
    ((SELECT IdIngrediente FROM Ingrediente WHERE Nome = 'Pepe nero'), 
     (SELECT IdRicetta FROM Ricetta WHERE Nome = 'Carbonara'), 2);
\end{lstlisting}
Questa query inserisce la composizione delle ricette nel database, specificando la ricetta, l'ingrediente e la quantità necessaria per ciascuno di essi. Per brevità abbiamo omesso parte del codice.\\\\

    \subsection{Query di inserimento dei topic}
    \lstinputlisting[caption=Query di inserimento dei topic]{../SQL/Insert/InsertTopic.sql}
    Questa query inserisce i topic nel database, specificando il nome e la descrizione del topic. I topic rappresentano le categorie tematiche dei corsi di cucina.\\\\

    \subsection{Query di inserimento della tabella insegna}
    \lstinputlisting[caption=Query di inserimento della tabella insegna]{../SQL/Insert/InsertInsegna.sql}
    Questa query inserisce i record nella tabella \textbf{Insegna}, che rappresenta la relazione tra i topic e gli chef che li insegnano.

    \subsection{Query di inserimento della tabella prepara}
    \lstinputlisting[caption=Query di inserimento della tabella prepara]{../SQL/Insert/InsertPrepara.sql}
    Questa query inserisce i record nella tabella \textbf{Prepara}, che rappresenta la relazione tra le sessioni in presenza e le ricette che vengono preparate durante le sessioni.

    \subsection{Query di inserimento delle iscrizioni}
    \lstinputlisting[caption=Query di inserimento delle iscrizioni]{../SQL/Insert/InsertIscrizione.sql}
    Questa query inserisce le iscrizioni degli utenti ai corsi di cucina, specificando l'utente, il corso e la data di iscrizione.
\end{document}