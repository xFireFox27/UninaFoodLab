\documentclass{book}
\usepackage[margin=3cm]{geometry}
\usepackage{graphicx} % Per inserire immagini

% STEP 1: include the package
\usepackage{uninafrontespizio}

% STEP 2: Front-page configuration
\Universita{Università degli Studi di Napoli Federico II}
\Facolta{Scuola Politecnica e delle Scienze di Base}
\Dipartimento{Dipartimento di Ingegneria Elettrica e Tecnologie dell'Informazione}
\CorsoDiLaurea{Corso di Laurea Triennale in Informatica}
% \Materia{Tesi di Laurea in Hacky Typesetting} % optional
\AnnoAccademico{Anno Accademico 2024--2025}
\Titolo{UninaFoodLab Database}

\Logo{logo-federico-II.pdf} %path to logo image
\LogoWidth{3.5cm} %optional, default: 3cm
\LogoPosition{below-uni} % or top, or below-title, or above-title, or no-logo

\begin{document}
    % STEP 3: use \makefrontpage and/or \makefrontpagealt
    \pagestyle{empty}
    \makefrontpage%
    % \makefrontpagealt
    \pagestyle{headings}
    \tableofcontents

    \chapter{Descrizione del progetto}
    \section{Descrizione sintetica del problema}
    Si vuole realizzare una base di dati relazionale per la gestione di corsi di cucina tematici.
    Gli chef possono registrare corsi su specifici argomenti, specificando una data di inizio e
    una frequenza delle sessioni. Ogni corso è articolato in più sessioni,
    che possono essere di due tipi: online, oppure in presenza, in cui gli utenti svolgono attività pratiche.
    Gli utenti possono iscriversi a più corsi e, nel caso delle sessioni pratiche, devono fornire una adesione
    esplicita per confermare la loro partecipazione. Ogni sessione pratica prevede la preparazione di una o
    più ricette, ciascuna delle quali richiede una specifica lista di ingredienti. Le adesioni vengono utilizzate
    per pianificare correttamente la quantità di ingredienti necessari, evitando così sprechi alimentari. 
    Il sistema prevede la possibilità di notificare gli utenti iscritti ad
    un corso in caso di variazioni di programma.
    
    \chapter{Progettazione Concettuale}
    \section{Introduzione}
    Una volta definito e analizzato il problema possiamo procedere con la progettazione concettuale della base di dati. In questa fase, 
    l'obiettivo è quello di creare un modello concettuale che rappresenti le entità, le relazioni e le caratteristiche del dominio. Tale schema viene
    rappresentato mediante un diagramma delle classi UML e un diagramma Entità-Relazioni (ER).    
    \section{Diagrammi UML e ER}
    \begin{figure}[h]
        \centering
        \includegraphics[width=\textwidth]{../Diagramma UML/UML.png}
        \caption{Diagramma delle classi UML}
        \label{fig:uml}
    \end{figure}
    \begin{figure}[h]
        \centering
        \includegraphics[width=\textwidth]{../Diagramma ER/ER.png}
        \caption{Diagramma Entità-Relazioni}
        \label{fig:er}
    \end{figure}
    \section{Ristrutturazione del modello concettuale}
    Un'operazione fondamentale per il passaggio dal modello concettuale a quello logico è la ristrutturazione. Questo processo 
    implica la trasformazione delle entità e delle relazioni del modello concettuale, al fine di renderne più agevole l'implementazione 
    in un database relazionale. Durante questa fase, si identificano le chiavi primarie e le chiavi esterne, si rimuovono gli attributi multipli
    e le gerarchie di specializzazione, si normalizzano le tabelle per ridurre la ridondanza dei dati.
    \section{Identificazione delle chiavi primarie}
    Le chiavi primarie sono gli attributi che identificano in modo univoco ogni record in una tabella. Nella nostra ristrutturazione
    abbiamo fatto uso di chiavi surrogate, ossia chiavi artificiali create per garantire l'unicità dei record nel caso in cui non esistano 
    chiavi naturali adatte. Queste chiavi non hanno un significato intrinseco, ma sono utilizzate per facilitare le operazioni sulle tabelle.
    \section{Identificazione delle chiavi esterne}
    Le chiavi esterne sono gli attributi che collegano le tabelle tra loro, rappresentando le relazioni tra le entità. Il loro scopo 
    è quello di assicurare che i dati siano coerenti e che le relazioni siano mantenute.
    \section{Rimozione degli attributi multipli}
    Non sono presenti attributi multipli nel modello concettuale.
    \section{Rimozione delle gerarchie di specializzazione}
    Sono presenti due gerarchie di specializzazione:
    \begin{itemize}
        \item 'Persona' si specializza in "Chef" e "Utente"
        \item 'Sessione' si specializza in "Sessione Online" e "Sessione in Presenza"
    \end{itemize}
    \section{Analisi delle ridondanze}
    Non sono presenti ridondanze significative nel modello concettuale.
    \section{Class Diagram ristrutturato}
    \begin{figure}[h]
        \centering
        \includegraphics[width=\textwidth]{../Diagramma UML Ristrutturato/UMLRistrutturato.png}
        \caption{Diagramma UML ristrutturato}
        \label{fig:uml-ristrutturato}
    \end{figure}
    
    \chapter{Progettazione Logica}
    \section{Schema Logico}
    \textbf{TOPIC}(\underline{IdTopic}, Nome, Descrizione)\\\\
    \textbf{INSEGNA}(\underline{\underline{IdTopic}}, \underline{\underline{UsernameChef}})\\\\
    \textbf{CORSO}(\underline{IdCorso}, Titolo, Frequenza, NumLezioni, DataInizio, \underline{\underline{IdTopic}}, \underline{\underline{UsernameChef}})\\\\
    \textbf{CHEF}(\underline{Username}, Nome, Cognome, Password)\\\\
    \textbf{UTENTE}(\underline{Username}, Nome, Cognome, Password)\\\\
    \textbf{NOTIFICA}(\underline{IdNotifica}, DataInvio, Oggetto, Testo, \underline{\underline{UsernameChef}})\\\\
    \textbf{RICEVE}(\underline{\underline{IdNotifica}}, \underline{\underline{UsernameUtente}})\\\\
    \textbf{ISCRIZIONE}(\underline{\underline{UsernameUtente}}, \underline{\underline{IdCorso}}, DataIscrizione)\\\\
    \textbf{SESSIONE ONLINE}(\underline{IdSessione}, Link, Data, Durata, NumSessione, \underline{\underline{IdCorso}})\\\\
    \textbf{SESSIONE IN PRESENZA}(\underline{IdSessione}, Luogo, Aula, Data, Durata, NumSessione, \underline{\underline{IdCorso}})\\\\
    \textbf{ADESIONE}(\underline{\underline{UsernameUtente}}, \underline{\underline{IdSessionePresenza}})\\\\
    \textbf{RICETTA}(\underline{IdRicetta}, Nome, Descrizione)\\\\
    \textbf{PREPARA}(\underline{\underline{IdSessionePresenza}}, \underline{\underline{IdRicetta}})\\\\
    \textbf{INGREDIENTE}(\underline{IdIngrediente}, Nome, UnitàDiMisura, Allergene)\\\\
    \textbf{COMPOSIZIONE}(\underline{\underline{IdRicetta}}, \underline{\underline{IdIngrediente}}, Quantità)\\\\
    \chapter{Progettazione Fisica}
    \section{Definizione tabelle}
   
\end{document}