\documentclass{book}
\usepackage[margin=3cm]{geometry}
\usepackage{graphicx} % Per inserire immagini

% STEP 1: include the package
\usepackage{uninafrontespizio}

% STEP 2: Front-page configuration
\Universita{Università degli Studi di Napoli Federico II}
\Facolta{Scuola Politecnica e delle Scienze di Base}
\Dipartimento{Dipartimento di Ingegneria Elettrica e Tecnologie dell'Informazione}
\CorsoDiLaurea{Corso di Laurea Triennale in Informatica}
% \Materia{Tesi di Laurea in Hacky Typesetting} % optional
\AnnoAccademico{Anno Accademico 2024--2025}
\Titolo{UninaFoodLab Database}


\Logo{logo-federico-II.pdf} %path to logo image
\LogoWidth{3.5cm} %optional, default: 3cm
\LogoPosition{below-uni} % or top, or below-title, or above-title, or no-logo

\begin{document}
    % STEP 3: use \makefrontpage and/or \makefrontpagealt
    \pagestyle{empty}
    \makefrontpage%
    % \makefrontpagealt
    \pagestyle{headings}
    \tableofcontents
    \chapter{Descrizione del progetto}
    \section{Descrizione sintetica del problema}
    \chapter{Progettazione Concettuale}
    \section{Introduzione}
    Una volta definito e analizzato il problema possiamo procedere con la progettazione concettuale della base di dati. In questa fase, 
    l'obiettivo è quello di creare un modello concettuale che rappresenti le entità, le relazioni e le caratteristiche del dominio. Tale schema viene
    rappresentato mediante un diagramma delle classi UML e un diagramma Entità-Relazioni (ER).    
    \section{Diagrammi}
    \begin{figure}[h]
        \centering
        \includegraphics[width=\textwidth]{../Diagramma UML/UML.png}
        \caption{Diagramma delle classi UML}
        \label{fig:uml}
    \end{figure}
    \begin{figure}[h]
        \centering
        \includegraphics[width=\textwidth]{../Diagramma ER/ER.png}
        \caption{Diagramma Entità-Relazioni}
        \label{fig:er}
    \end{figure}
    
    \chapter{Progettazione Logica}
    \section{Schema Logico}
    \chapter{Progettazione Fisica}
    \section{Definizione tabelle}
   
\end{document}