\documentclass[oneside]{book}
\usepackage[margin=3cm]{geometry}
\usepackage[utf8]{inputenc}
\usepackage[T1]{fontenc}
\usepackage[italian]{babel}
\usepackage{graphicx}
\usepackage{listings}
\usepackage{xcolor}
\usepackage{uninafrontespizio}




% Configurazione SQL
\lstset{
    language=Java,
    basicstyle=\ttfamily\small,
    keywordstyle=\color{blue}\bfseries,
    commentstyle=\color{green!60!black},
    stringstyle=\color{red},
    numberstyle=\tiny\color{gray},
    numbers=left,
    numbersep=5pt,
    frame=single,
    breaklines=true,
    breakatwhitespace=true,
    tabsize=2,
    showspaces=false,
    showstringspaces=false,
    captionpos=b,
    literate={à}{{\`a}}1 {è}{{\`e}}1 {ì}{{\`i}}1 {ò}{{\`o}}1 {ù}{{\`u}}1
             {À}{{\`A}}1 {È}{{\`E}}1 {Ì}{{\`I}}1 {Ò}{{\`O}}1 {Ù}{{\`U}}1
             {á}{{\'a}}1 {é}{{\'e}}1 {í}{{\'i}}1 {ó}{{\'o}}1 {ú}{{\'u}}1
             {Á}{{\'A}}1 {É}{{\'E}}1 {Í}{{\'I}}1 {Ó}{{\'O}}1 {Ú}{{\'U}}1
}

\usepackage{uninafrontespizio}

\Universita{Università degli Studi di Napoli Federico II}
\Facolta{Scuola Politecnica e delle Scienze di Base}
\Dipartimento{Dipartimento di Ingegneria Elettrica e Tecnologie dell'Informazione}
\CorsoDiLaurea{Corso di Laurea Triennale in Informatica}
\Materia{Programmazione Object Oriented}
\AnnoAccademico{Anno Accademico 2024--2025}
\Titolo{UninaFoodLab}

\Logo{logo-federico-II.pdf}
\LogoWidth{3.5cm} 
\LogoPosition{below-uni}

\begin{document}
    \pagestyle{empty}
    \makefrontpage
    \pagestyle{plain}
    \tableofcontents

    \chapter{Descrizione del progetto}
        \section{Descrizione sintentica del problema}
            Si sviluppi un applicativo Java con interfaccia grafica per la gestione dei corsi tematici 
            offerti dalla piattaforma UninaFoodLab. Il sistema dovrà essere collegato a un database relazionale prepopolato
            contenente informazioni su chef, ricette e ingredienti. Il sistema deve permettere
            l’autenticazione degli chef tramite credenziali (username e password). Una volta autenticato, lo chef 
            può aggiungere un nuovo corso, specificando le seguenti informazioni: categoria, data di inizio, 
            frequenza delle sessioni, numero di sessioni. Per ciascuna sessione, deve essere indicata la modalità di 
            svolgimento, ovvero se si tratta di una sessione online o in presenza. Lo chef avrà inoltre la possibilità 
            di visualizzare i corsi esistenti, applicando filtri per categoria. Dopo aver selezionato un corso, lo chef 
            può associare a ciascuna sessione pratica una o più ricette da realizzare. Infine, il sistema deve fornire 
            un report mensile, che permette allo chef di visualizzare: il numero di corsi totali tenuti, il numero di 
            sessioni online e pratiche, e di quest’ultime il numero medio, massimo e minimo di ricette realizzate. Il 
            report deve fornire una rappresentazione grafica dei dati. Il sistema deve permettere allo chef di inserire delle
            notifiche relative ai propri corsi, in caso di modifiche, come cambio di data o ora di una sessione oppure la sua 
            cancellazione. Durante la creazione della notifica, lo chef specifica se è destinata ad un singolo corso 
            oppure se è risolva a tutti. Tutte le notifiche sono consultabili in un’apposita sezione dell’interfaccia. 

    \chapter{Diagramma delle Classi}

    \section{UML}
            \begin{figure}[h]
                \centering
                \includegraphics[width=\textwidth]{UninaFoodLabApp.png}
                \caption{Diagramma delle classi UML}
            \end{figure}
        \section{Architettura}
            L'Architettura dell'applicativo è basata sul pattern EBC (Entity Boundary Control). Questo pattern prevede la separazione delle responsabilità tra le diverse componenti del sistema, facilitando la manutenibilità e l'estensibilità dell'applicativo.
            Come si può vedere dal diagramma UML, le classi sono suddivise in tre categorie principali con ognuna di esse che svolge un ruolo specifico: 
            \begin{itemize}
                \item \textbf{Entity}: Queste classi rappresentano le entità principali del dominio dell'applicativo, come Chef, Corso, Sessione, Ricetta, Ingrediente e Notifica. Esse contengono gli attributi e i metodi necessari per gestire i dati relativi a queste entità.
                \item \textbf{Boundary}: Le classi di questa categoria sono responsabili dell'interfaccia utente e della gestione delle interazioni con l'utente. Ad esempio, la classe UninaFoodLabApp funge da punto di ingresso per l'applicativo, mentre altre classi come LoginFrame, MainFrame e CourseManagementFrame gestiscono le diverse schermate dell'interfaccia grafica.
                \item \textbf{Control}: Queste classi agiscono come intermediari tra le classi Entity e Boundary. Gestiscono la logica di business dell'applicativo, coordinando le operazioni tra le entità e l'interfaccia utente. Ad esempio, la classe CourseController gestisce le operazioni relative ai corsi, mentre NotificationController si occupa della gestione delle notifiche.
            \end{itemize}
            Per l'interfaccia grafica è stato utilizzato il framework Swing di Java, che offre una vasta gamma di componenti per la creazione di interfacce utente ricche e interattive. Le classi del package Boundary estendono JFrame e utilizzano vari componenti Swing come JButton, JTextField, JComboBox e JTable per costruire le diverse schermate dell'applicativo.
            La connessione al database è gestita tramite JDBC (Java Database Connectivity), che consente di eseguire query SQL e di interagire con il database relazionale.
            
    
<<<<<<< HEAD
    \chapter{Descrizione Package e classi}
        
        \section{Package \textit{Boundary}}
            Il pacchetto boundary contiene tutte le classi che regolano i Frame, la loro impostazione grafica,
            con disposizione dei widget e il loro funzionamento.
            \subsection{Classe LoginFrame}
                Questo è il frame per effettuare il login come chef nell'applicazione inserendo username e password validi.
            \subsection{Classe HomepageChef}
                Questo è il frame dell'HomePage dello chef dalla quale è possibile selezionare, grazie agli appositi bottoni,
                quale schermata visualizzare tra: Creazione corsi, Gestione Corsi, Notifiche e Riepilogo Mensile.
            \subsection{Classe NuovoCorsoFrame}
                Questo è il frame che permette la creazione di un nuovo corso
            \subsection{Classe CorsiFrame}
                Questo è il frame da cui è possibile visualizzare tutti i corsi dello chef loggato in formato di lista
                e filtrarli per Topic. Facendo doppio click sul nome di un corso è inoltre possibile aprire la schermata
                per l'inserimento di sessioni.
            \subsection{Classe SessioniFrame}
                Questo è il frame che permette di visualizzare e gestire le sessioni. Ogni sessione è selezionabile da
                una lista, selezionando una sessione in presenza verrà aperto il frame per la gestione delle ricette per quella
                sessione.
            \subsection{Classe InserimentoSessioneFrame}
                Questo è il frame che permette di creare le sessioni per il corso selezionato.
            \subsection{Classe RicetteFrame}
                Questo è il frame per la gestione delle ricette di una sessione in presenza, ad ogni sessione possono essere
                associate diverse ricette attraverso una checklist. Se una sessione ha già dele ricette associate un dialog chiede
                se si vuole che queste vengano sovrascritte.
            \subsection{Classe NotificheFrame}
                Questo è il frame dal quale è possibile visualizzare tutte le notifiche inviate dallo chef loggato
                e, attraverso l'apposito bottone, aprire il frame per l'invio di una nuova notifica.
            \subsection{Classe InvioNotificheFrame}
                Questo è il frame dal quale è possibile inviare nuove notifiche ad un corso specifico, selezionabile tramite ComboBox,
                oppure a tutti i corsi dello chef loggato.
            \subsection{Classe RiepilogoMensileFrame}
                Questo è il frame dal quale è possibile visualizzare il riepilogo mensile contenente: il numero di corsi totali tenuti,
                il numero di sessioni online e pratiche, e di quest’ultime il numero medio, massimo e minimo di ricette realizzate,
                sottoforma di diagramma a barre.
=======
    \chapter{Descrizione dei Package e delle Classi}
        \section{Descrizione del package Boundary}
            Il pacchetto contiene le classi della gui
>>>>>>> origin/main
        \section{Package \textit{Control}}
            Il pacchetto contiene la classe controller che gestisce la logica dell'applicativo determinando le interazioni tra le classi e la gestione delle risorse. 
            \subsection{Classe Controller}
                Il controller, seguendo gli schemi dell'architettura BCE, si occupa di gestire le interazioni tra le interfacce del programma e le entità del dominio. Di conseguenza il controller ha il compito 
                di amministrare l'apertura e la chiusura delle finestre della GUI, di interpretare l'interazione tra l'utente e gli elementi dell'interfaccia grafica, di gestire l'allocazione e la liberazione
                di risorse interagendo con il database mediante l'utilizzo delle classi DAO.
        \section{Package \textit{Entity}}
            Il pacchetto contiene le classi che definiscono le entità del dominio.
            \subsection{Classe Chef}
                La classe Chef rappresenta un utente del sistema che può autenticarsi, gestire i propri corsi, creandone di nuovi o visualizzando quelli esistenti, e inserire notifiche relative ai corsi da lui tenuti.
                Lo Chef può, inoltre, gestire le sessioni dei propri corsi e le ricette delle sessioni in presenza.
            \subsection{Classe Corso}
                La classe Corso rappresenta i corsi tenuti dagli chef. Il Corso è caratterizzato da un titolo, una frequenza delle lezioni, una data di inizio, un anno in cui è tenuto il corso, un numero di 
                lezioni, un topic e uno chef responsabile del corso.
            \subsection{Classe Notifica}
                La classe Notifica rappresenta le notifiche inviate dagli chef a corsi specifici e ricevute dagli studenti degli stessi corsi. La Notifica è caratterizzata da un oggetto, un testo e una data di invio.
            \subsection{Classe Prepara}
                La classe Prepara rappresenta la relazione tra le ricette e le sessioni in presenza. Di conseguenza gli attributi della classe sono quelli relativi alle ricette e alle sessioni associate.
            \subsection{Classe Ricetta}
                La classe Ricetta rappresenta le ricette che possono essere preparate durante le sessioni in presenza. La Ricetta è caratterizzata da un nome e una descrizione.
            \subsection{Classe Sessione}
                La classe Sessione rappresenta le lezioni, che compongono un corso. La Sessione è caratterizzata da una data, una durata, un numero e un corso di appartenenza. 
            \subsection{Classe SessioneInPresenza}
                La classe SessioneInPresenza estende la classe Sessione e rappresenta le sessioni che si svolgono in presenza. La SessioneInPresenza è caratterizzata da una sede in cui si svolge la lezione e un'aula.
            \subsection{Classe SessioneOnline}
                La classe SessioneOnline estende la classe Sessione e rappresenta le sessioni che si svolgono online. La SessioneOnline è caratterizzata da un link di accesso.
            \subsection{Classe Topic}
                La classe Topic rappresenta le categorie dei corsi e le competenze degli chef. Il Topic è caratterizzato da un nome e una descrizione.



        \section{Package \textit{DaoInterface}}
            Il package contiene le interfacce che definiscono i metodi per l'accesso ai dati nel database. Inoltre nel caso in cui si volesse cambiare il database, basterebbe creare una nuova classe DAO che implementa queste interfacce.
        \section{Package \textit{DAO}}
            Il package contiene le classi che si occupano della connessione al database e dell'esecuzione delle query SQL.
            \subsection{Classe ChefDAO}
                La classe ChefDAO si occupa della gestione delle operazioni relative alla tabella Chef del database.\\
                Tramite il metodo creaChef di questa classe, è possibile istanziare un oggetto Chef a partire  dai dati presenti nel database ed effettuare il login nell'applicativo.
            \subsection{Classe CorsoDAO}
                La classe CorsoDAO si occupa della gestione delle operazioni relative alla tabella Corso del database.\\
                Tramite i metodi di questa classe, è possibile aggiungere un nuovo corso, recuperare i corsi esistenti e applicare filtri per categoria.
            \subsection{Classe NoitificaDAO}
                La classe NotificaDAO si occupa della gestione delle operazioni relative alla tabella Notifica del database.\\
                Tramite i metodi di questa classe, è possibile aggiungere una nuova notifica, recuperare le notifiche esistenti e visualizzarle nell'apposita sezione dell'interfaccia.
            \subsection{Classe PreparaDAO}
                La classe PreparaDAO si occupa della gestione delle operazioni relative alla tabella Prepara del database.\\
                Tramite i metodi di questa classe, è possibile associare una o più ricette a ciascuna sessione pratica di un corso. 
            \subsection{Classe RicettaDAO}
                La classe RicettaDAO si occupa della gestione delle operazioni relative alla tabella Ricetta del database.\\
                Tramite i metodi di questa classe, è possibile recuperare le ricette esistenti e visualizzarle nell'apposita sezione dell'interfaccia.
            \subsection{Classe SessioneOnlineDAO}
                La classe SessioneOnlineDAO si occupa della gestione delle operazioni relative alla tabella SessioneOnline del database.\\
                Tramite i metodi di questa classe, è possibile aggiungere una nuova sessione online o recuperare le sessioni online esistenti e visualizzarle nell'apposita sezione dell'interfaccia.
            \subsection{Classe SessioneInPresenzaDAO}
                La classe SessioneInPresenzaDAO si occupa della gestione delle operazioni relative alla tabella SessioneInPresenza del database.\\
                Tramite i metodi di questa classe, è possibile aggiungere una nuova sessione in presenza o recuperare le sessioni in presenza esistenti e visualizzarle nell'apposita sezione dell'interfaccia.
            \subsection{Classe TopicDAO}
                La classe TopicDAO si occupa della gestione delle operazioni relative alla tabella Topic del database.\\
                Tramite i metodi di questa classe, è possibile recuperare i topic esistenti e visualizzarle nell'apposita sezione dell'interfaccia.
    \chapter{Funzionalità dell'applicativo}



\end{document}