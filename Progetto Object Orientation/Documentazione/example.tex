\documentclass[oneside]{book}
\usepackage[margin=3cm]{geometry}
\usepackage[utf8]{inputenc}
\usepackage[T1]{fontenc}
\usepackage[italian]{babel}
\usepackage{graphicx}
\usepackage{listings}
\usepackage{xcolor}




% Configurazione SQL
\lstset{
    language=Java,
    basicstyle=\ttfamily\small,
    keywordstyle=\color{blue}\bfseries,
    commentstyle=\color{green!60!black},
    stringstyle=\color{red},
    numberstyle=\tiny\color{gray},
    numbers=left,
    numbersep=5pt,
    frame=single,
    breaklines=true,
    breakatwhitespace=true,
    tabsize=2,
    showspaces=false,
    showstringspaces=false,
    captionpos=b,
    literate={à}{{\`a}}1 {è}{{\`e}}1 {ì}{{\`i}}1 {ò}{{\`o}}1 {ù}{{\`u}}1
             {À}{{\`A}}1 {È}{{\`E}}1 {Ì}{{\`I}}1 {Ò}{{\`O}}1 {Ù}{{\`U}}1
             {á}{{\'a}}1 {é}{{\'e}}1 {í}{{\'i}}1 {ó}{{\'o}}1 {ú}{{\'u}}1
             {Á}{{\'A}}1 {É}{{\'E}}1 {Í}{{\'I}}1 {Ó}{{\'O}}1 {Ú}{{\'U}}1
}

% STEP 1: include the package
\usepackage{uninafrontespizio}

% STEP 2: Front-page configuration
\Universita{Università degli Studi di Napoli Federico II}
\Facolta{Scuola Politecnica e delle Scienze di Base}
\Dipartimento{Dipartimento di Ingegneria Elettrica e Tecnologie dell'Informazione}
\CorsoDiLaurea{Corso di Laurea Triennale in Informatica}
% \Materia{Tesi di Laurea in Hacky Typesetting} % optional
\AnnoAccademico{Anno Accademico 2024--2025}
\Titolo{UninaFoodLab Database}

\Logo{logo-federico-II.pdf} %path to logo image
\LogoWidth{3.5cm} %optional, default: 3cm
\LogoPosition{below-uni} % or top, or below-title, or above-title, or no-logo

\begin{document}
    % STEP 3: use \makefrontpage and/or \makefrontpagealt
    \pagestyle{empty}
    \makefrontpage%
    % \makefrontpagealt
    \pagestyle{plain}
    \tableofcontents

    \chapter{Descrizione del progetto}
    Si sviluppi un applicativo Java con interfaccia grafica per la gestione dei corsi tematici 
    offerti dalla piattaforma UninaFoodLab. Il sistema dovrà essere collegato a un database relazionale prepopolato
    contenente informazioni su chef, ricette e ingredienti. Il sistema deve permettere
    l’autenticazione degli chef tramite credenziali (username e password). Una volta autenticato, lo chef 
    può aggiungere un nuovo corso, specificando le seguenti informazioni: categoria, data di inizio, 
    frequenza delle sessioni, numero di sessioni. Per ciascuna sessione, deve essere indicata la modalità di 
    svolgimento, ovvero se si tratta di una sessione online o in presenza. Lo chef avrà inoltre la possibilità 
    di visualizzare i corsi esistenti, applicando filtri per categoria. Dopo aver selezionato un corso, lo chef 
    può associare a ciascuna sessione pratica una o più ricette da realizzare. Infine, il sistema deve fornire 
    un report mensile, che permette allo chef di visualizzare: il numero di corsi totali tenuti, il numero di 
    sessioni online e pratiche, e di quest’ultime il numero medio, massimo e minimo di ricette realizzate. Il 
    report deve fornire una rappresentazione grafica dei dati. Il sistema deve permettere allo chef di inserire delle
    notifiche relative ai propri corsi, in caso di modifiche, come cambio di data o ora di una sessione oppure la sua 
    cancellazione. Durante la creazione della notifica, lo chef specifica se è destinata ad un singolo corso 
    oppure se è risolva a tutti. Tutte le notifiche sono consultabili in un’apposita sezione dell’interfaccia. 

\end{document}