\chapter{Repository GitHub}

\section{Cos'è una Repository}
Una repository è uno spazio di archiviazione digitale dove vengono conservati i file di un progetto software insieme alla loro cronologia completa delle modifiche. GitHub è una piattaforma web che ospita repository Git, offrendo strumenti collaborativi per lo sviluppo software.


\section{Struttura del Progetto}
Il progetto UninaFoodLab è organizzato in una struttura chiara che facilita la navigazione e la comprensione del codice:
\begin{itemize}
    \item \textbf{Codice sorgente}: Organizzato in package secondo il pattern architetturale EBC
    \item \textbf{Documentazione}: Include diagrammi UML, specifiche tecniche e guide utente
    \item \textbf{Database}: Script SQL per la creazione e il popolamento del database
    \item \textbf{Risorse}: Immagini, icone e altri file multimediali utilizzati dall'applicazione
\end{itemize}

\section{Accesso al Codice}
Il codice sorgente del progetto è disponibile al seguente indirizzo: 

\href{https://github.com/xFireFox27/UninaFoodLab.git}{https://github.com/xFireFox27/UninaFoodLab.git}

La repository contiene l'intero progetto con la cronologia completa delle modifiche, permettendo di seguire l'evoluzione del software durante lo sviluppo.