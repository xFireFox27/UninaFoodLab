           \section{Descrizione sintentica del problema}
           Si sviluppi un applicativo Java con interfaccia grafica per la gestione dei corsi tematici 
            offerti dalla piattaforma UninaFoodLab. Il sistema dovrà essere collegato a un database relazionale prepopolato
            contenente informazioni su chef, ricette e ingredienti. Il sistema deve permettere
            l’autenticazione degli chef tramite credenziali (username e password). Una volta autenticato, lo chef 
            può aggiungere un nuovo corso, specificando le seguenti informazioni: categoria, data di inizio, 
            frequenza delle sessioni, numero di sessioni. Per ciascuna sessione, deve essere indicata la modalità di 
            svolgimento, ovvero se si tratta di una sessione online o in presenza. Lo chef avrà inoltre la possibilità 
            di visualizzare i corsi esistenti, applicando filtri per categoria. Dopo aver selezionato un corso, lo chef 
            può associare a ciascuna sessione pratica una o più ricette da realizzare. Infine, il sistema deve fornire 
            un report mensile, che permette allo chef di visualizzare: il numero di corsi totali tenuti, il numero di 
            sessioni online e pratiche, e di quest’ultime il numero medio, massimo e minimo di ricette realizzate. Il 
            report deve fornire una rappresentazione grafica dei dati. Il sistema deve permettere allo chef di inserire delle
            notifiche relative ai propri corsi, in caso di modifiche, come cambio di data o ora di una sessione oppure la sua 
            cancellazione. Durante la creazione della notifica, lo chef specifica se è destinata ad un singolo corso 
            oppure se è risolva a tutti. Tutte le notifiche sono consultabili in un’apposita sezione dell’interfaccia.