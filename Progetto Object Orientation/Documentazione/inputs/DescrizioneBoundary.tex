
        \section{Package \textit{Boundary}}
            Il pacchetto boundary contiene tutte le classi che regolano i Frame, la loro impostazione grafica,
            con disposizione dei widget e il loro funzionamento.
            \subsection{Classe LoginFrame}
                Questo è il frame per effettuare il login come chef nell'applicazione inserendo username e password validi.
            \subsection{Classe HomepageChef}
                Questo è il frame dell'HomePage dello chef dalla quale è possibile selezionare, grazie agli appositi bottoni,
                quale schermata visualizzare tra: Creazione corsi, Gestione Corsi, Notifiche e Riepilogo Mensile.
            \subsection{Classe NuovoCorsoFrame}
                Questo è il frame che permette la creazione di un nuovo corso
            \subsection{Classe CorsiFrame}
                Questo è il frame da cui è possibile visualizzare tutti i corsi dello chef loggato in formato di lista
                e filtrarli per Topic. Facendo doppio click sul nome di un corso è inoltre possibile aprire la schermata
                per l'inserimento di sessioni.
            \subsection{Classe SessioniFrame}
                Questo è il frame che permette di visualizzare e gestire le sessioni. Ogni sessione è selezionabile da
                una lista, selezionando una sessione in presenza verrà aperto il frame per la gestione delle ricette per quella
                sessione.
            \subsection{Classe InserimentoSessioneFrame}
                Questo è il frame che permette di creare le sessioni per il corso selezionato.
            \subsection{Classe RicetteFrame}
                Questo è il frame per la gestione delle ricette di una sessione in presenza, ad ogni sessione possono essere
                associate diverse ricette attraverso una checklist. Se una sessione ha già dele ricette associate un dialog chiede
                se si vuole che queste vengano sovrascritte.
            \subsection{Classe NotificheFrame}
                Questo è il frame dal quale è possibile visualizzare tutte le notifiche inviate dallo chef loggato
                e, attraverso l'apposito bottone, aprire il frame per l'invio di una nuova notifica.
            \subsection{Classe InvioNotificheFrame}
                Questo è il frame dal quale è possibile inviare nuove notifiche ad un corso specifico, selezionabile tramite ComboBox,
                oppure a tutti i corsi dello chef loggato.
            \subsection{Classe RiepilogoMensileFrame}
                Questo è il frame dal quale è possibile visualizzare il riepilogo mensile contenente: il numero di corsi totali tenuti,
                il numero di sessioni online e pratiche, e di quest’ultime il numero medio, massimo e minimo di ricette realizzate,
                sottoforma di diagramma a barre.