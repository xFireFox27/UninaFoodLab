\section{I Miei Corsi}

\begin{figure}[h]
    \centering
    \includegraphics[width=0.8\textwidth]{Images/IMieiCorsi.png}
    \caption{Schermata I Miei Corsi}
\end{figure}

Questa sezione consente allo chef di visualizzare e gestire i corsi di cucina che ha creato.
Sulla destra della schermata, è presente un elenco di corsi creati dallo chef, con informazioni come il nome del corso, il numero delle lezioni, il topic e la frequenza.
Mentre sulla sinistra è presente un utile filtro di ricerca che permette di cercare corsi in base ad al topic selezionato.
Cliccando su un corso dall'elenco, lo chef può visualizzare le sessioni associate a quel corso.


    \newpage
    \subsection{Gestione Sessioni}

        \begin{figure}[h]
            \centering
            \includegraphics[width=0.8\textwidth]{Images/GestioneSessioni.png}
            \caption{Gestione Sessioni}
        \end{figure}

        In questa schermata, lo chef può visualizzare tutte le sessioni associate al corso selezionato.
        Le sessioni si dividono in sessioni in presenza e sessioni online.
        Per le sessioni in presenza vengono mostrate informazioni come la data, l'ora, la sede e l'aula.
        Per le sessioni online vengono mostrate informazioni come la data, l'ora e il link di accesso.
        Lo chef selezionando una sessione in presenza può accedere alla gestione delle ricette per quella sessione.
        Inoltre, cliccando il pulsante "Nuova Sessione", lo chef può inserire una nuova sessione per il corso selezionato.
    \newpage
    \subsection{Inserimento Nuova Sessione}

        \begin{figure}[h]
            \centering
            \includegraphics[width=0.8\textwidth]{Images/NuovaSessione.png}
            \caption{Inserimento Nuova Sessione}
        \end{figure}

        In questa schermata, lo chef può inserire i dettagli per una nuova sessione del corso selezionato.
        Deve specificare prima il tipo della sessione (in presenza o online).
        In base alla scelta, deve inserire le informazioni richieste come la data, l'ora, la sede e l'aula per le sessioni in presenza, oppure il link di accesso per le sessioni online.
        Inoltre è necessaria la durata della sessione e il numero di quest'ultima.
        Una volta inseriti tutti i dettagli, lo chef può premere il pulsante "Salva" per creare la nuova sessione e aggiungerla all'elenco.
    \newpage
    \subsection{Gestione Ricette}

        \begin{figure}[h]
            \centering
            \includegraphics[width=0.8\textwidth]{Images/GestioneRicette.png}
            \caption{Gestione Ricette}
        \end{figure}

        In questa schermata, lo chef può visualizzare e gestire le ricette da associare alle sessioni in presenza.
        Lo chef può vedere l'elenco delle ricette e può scegliere di associarle alla sessione selezionata spèuntando quelle che preferisce.
        Una volta selezionate, lo chef può premere il pulsante "Salva" per confermare l'associazione delle ricette alla sessione.
        In caso una sessione gia abbia delle ricette associate, quando si prova a salvare delle ricette verrà mostrato un messaggio di conferma sovrascrittura.

        \begin{figure}[h]
            \centering
            \includegraphics[width=0.8\textwidth]{Images/SovrascrizioneRicette.png}
            \caption{Conferma Sovrascrittura Ricette}
        \end{figure}
    